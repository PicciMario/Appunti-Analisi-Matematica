\documentclass[a4paper,10pt,italian]{article}
\textwidth 460pt
\textheight 700pt
\hoffset -60pt
\voffset -80pt
\usepackage[latin1]{inputenc}
\usepackage{babel}
\usepackage[dvips]{graphicx}
\usepackage{textcomp}
\usepackage{amsmath}

\title{Formulario Analisi C}
\author{bug report: mario.piccinelli@gmail.com}
\date{}

\begin{document}

\maketitle

This work is licensed under the Creative Commons Attribution-Noncommercial-Share Alike 2.5 Italy License. To view a copy of this license, visit http://creativecommons.org/licenses/by-nc-sa/2.5/it/ or send a letter to Creative Commons, 171 Second Street, Suite 300, San Francisco, California, 94105, USA.

\section{Serie note}
\begin{itemize}
\item Serie geometrica
$$ \sum_{n=0}^{\infty} q^n \quad = \quad \frac{1}{1-q} $$
converge se $\mid q \mid < 1$
\item Serie armonica generalizzata
$$ \sum_{n=1}^{+\infty} \Bigr( \frac{1}{n} \Bigr) ^\alpha $$
converge se $\alpha < 1$
\item Serie
$$ \sum_{n=2}^{+\infty} \frac{1}{n^\alpha \cdot (log \, n)^\beta} $$ 
converge se $\alpha > 1, \forall \beta$ oppure $\alpha=1, \beta>1$
\end{itemize}

\section{Sviluppi di Taylor}
\begin{align*}
\frac{1}{1-x} \quad &= \quad \sum_{n=0}^{+\infty}x^n &\mbox{per }\mid x \mid < 1 \\
e^x \quad &= \quad \sum_{n=0}^{+\infty}\frac{x^n}{n!} \\
log(1+x) \quad &= \quad \sum_{n=1}^{+\infty}(-1)^{n+1}\cdot\frac{x^n}{n} &\mbox{per } -1<x\leq 1 \\
sin(x) \quad &= \quad \sum_{n=0}^{+\infty}(-1)^n\cdot\frac{x^{2n+1}}{(2n+1)!} \\
cos(x) \quad &= \quad \sum_{n=0}^{+\infty}(-1)^n\cdot\frac{x^{2n}}{(2n)!} \\
sinh(x) \quad &= \quad \sum_{n=0}^{+\infty}\frac{x^{2n+1}}{(2n+1)!} \\
cosh(x) \quad &= \quad \sum_{n=0}^{+\infty}\frac{x^{2n}}{(2n)!} \\
arctan(x) \quad &= \quad \sum_{n=0}^{+\infty}(-1)^n\cdot\frac{x^{2n+1}}{2n+1} &\mbox{per }\mid x \mid \leq 1\\
arctanh(x) \quad &= \quad \sum_{n=0}^{+\infty}\frac{x^{2n+1}}{2n+1} \\
\end{align*}

\newpage

\section{Integrali impropri notevoli}
\begin{align*}
&\int_{0}^{1} \Bigr( \frac{1}{x} \Bigr) ^\alpha & \mbox{converge se } \alpha < 1 \\
&\int_{1}^{+\infty} \Bigr( \frac{1}{x} \Bigr) ^\alpha & \mbox{converge se } \alpha > 1 \\
&\int_{0}^{\frac{1}{2}} \frac{1}{x^\alpha \cdot \mid log\,x \mid ^\beta}\,dx & \mbox{converge se } \alpha < 1 \quad \forall \beta \\
&& \mbox{oppure } \alpha = 1 \quad \beta > 1 \\
&\int_{a>1}^{+\infty} \frac{1}{x^\alpha \cdot \mid log\,x \mid ^\beta}\,dx & \mbox{converge se } \alpha > 1 \quad \forall \beta \\
&& \mbox{oppure } \alpha = 1 \quad \beta > 1 \\
\end{align*}

%\huge{CONTINUA}...


%fine documento
\end{document}
